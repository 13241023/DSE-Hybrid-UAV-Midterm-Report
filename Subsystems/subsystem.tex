









\begin{comment}
\section{The Airplane Balloon}
This design will be deleted :( :(((


\subsection{Aerodynamic Subsystems}

\subsubsection{Lift surfaces}

\subsubsection*{Balloon}

The balloon will have a major effect on the aerodynamics of the plane. During take-off, landing and hovering the wind will blow onto the major size of the balloon and pushing it aside. This will make it difficult to stay in place, and may float away rather easy. To counteract this, constant measures will be needed, as some force against the direction of the wind. The side of the balloon will have a greater surface area than the front of the balloon due to the shape of the plane, the x-axis of the plane will align automatically with the wind.

When accelerating to be able to fly without balloon, the balloon will generate a considerable amount of drag. This drag will decrease once the balloon starts to deflate, but at the start (when there is no horizontal speed) the lift is fully generated by the buoyancy of the balloon, hence the balloon will be at its biggest and generate a lot of drag. This means that a considerable amount of thrust is needed at the start.

\subsubsection*{Wing}

The weight of this aircraft will probably be substantial, this means a lot of lift will be needed, which in turn means that the wings need to be large. There will be some wing-propeller interference depending on where the propellers are placed. Having the propellers placed on the leading edge will increase lift because the propeller will increase the flow over the wings. The airfoil for the wing would be a conventional airfoil as found on airplanes, which focuses on high performances for cruise flight.

\subsubsection*{Tail plane}

For an aerodynamic point of view, the tailplane should create as low drag as possible. Due to the wing generating downwash, it would not be the best idea to put the whole tailplane in the downwash of the wing. Having the tailplane on top would benefit the aerodynamic capability of the tailplane.
%deep stall may be considered in control


\subsection{Fuselage}

The best option for the fuselage will be to have the same configuration as a conventional airplane, it will be cylindrical and have a cone-shaped nose. This way the fuselage will generate low drag. The fuselage will have at least front surface area of 15x15 cm due to the payload restriction, but due to the balloon needing to be fit inside the fuselage, it will probably be bigger. The bigger the fuselage, the more drag it will create during flight. Not only more drag will be created, but the dynamic stability may decrease with a higher diameter of the aircraft due to the aerodynamic forces acting on the fuselage (it will be destabilising). 


\subsection{Power \& Propulsion Subsystems}
\subsection{Structural Subsystems}



\subsection{Stability and Control Subsystems}
The subsystems for stability and control are divided between the horizontal high velocity flight phase and the vertical phases consisting of hovering, vertical take-off and vertical landing.


\subsubsection{Horizontal Flight Phases}
During horizontal flight, the balloon is deflated. Therefore, the balloon can not be used for stability in horizontal flight. For pitch control, elevators are needed. In order to control the roll rate, ailerons should be used. Yaw control can be achieved using a rudder and differential thrust on the propellers, or both. 



\subsubsection{Vertical Flight Phases}
In order to have control of the UAV during the vertical flight phases, two subsystems are required. The first one is used to control the vertical velocity and the height, as well as the pitch and roll moments. This is done by splitting the balloon up in multiple compartments and enabling in- and deflation of separate compartments. The attitude shall be actively controlled by a computer, ensuring that the UAV is stable.
The yaw will be controlled using the propellers that are also used for horizontal control. Here it is required that the propellers are also able to produce thrust in the reverse direction.
%stability using the balloon is very slow - take this into account at trade off






\section{The Tandem}

\subsection{Aerodynamic Subsystems}
\subsection{Power \& Propulsion Subsystems}
\subsection{Structural Subsystems}




\subsection{Stability and Control Subsystems}



\subsubsection{Horizontal Flight Phases}


\subsubsection{Vertical }





\section{The Prandtl Box}

\subsection{Aerodynamic Subsystems}
\subsection{Power \& Propulsion Subsystems}
\subsection{Structural Subsystems}
\subsection{Stability and Control Subsystems}



\section{The Flying Ducted Wing}

\subsection{Aerodynamic Subsystems}
\subsection{Power \& Propulsion Subsystems}
\subsection{Structural Subsystems}
\subsection{Stability and Control Subsystems}


\section{The Tilt-Rotor}

\subsection{Aerodynamic Subsystems}
\subsection{Power \& Propulsion Subsystems}
\subsection{Structural Subsystems}
\subsection{Stability and Control Subsystems}

\end{comment}


