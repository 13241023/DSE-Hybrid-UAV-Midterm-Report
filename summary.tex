\chapter*{Summary} %PLEASE STOP CHANGING THIS

%Outline of background (why are we writing this)
\begin{comment}
- UAV market is growing
- Dull, dirty, dangerous missions
New technologies such as improved Lithium batteries allow the UAVs to be developed further and further. Since a large amount of missions is either dull, dirty or dangerous, a demand for drones makes sense. Such boring or life-threatening missions, that have been performed using manned aircraft in the past, are a great opportunity for UAVs.  
\end{comment}

The Unmanned Arial Vehicle (UAV) market is growing\footnotemark, as the field of application becomes increasingly diverse \cite{baseline}. New technologies such as improved lithium batteries allow the UAVs to be designed lighter and smaller, or to increase their range. A large amount of missions is either dull, dirty or dangerous, and are therefore better executed by UAVs instead of manned aircraft. 

\footnotetext{\url{http://www.businessinsider.com/uav-or-commercial-drone-market-forecast-2015-2?international=true&r=US&IR=T}, Accessed 22-05-2017} 


The spectrum of missions that can be performed by UAVs is broad. On the one end are small and inexpensive UAVs that are used by consumers and companies alike. On the other end are very large military UAVs, which have an intercontinental range and endurance in the order of days. Many missions require both operation without extensive facilities, while also achieving high-speed flight. This is why a Hybrid UAV would be very useful: it could be operated from virtually anywhere as it does not require a runway, but has the flight performance of a conventional aircraft.

%Purpose of the report
\begin{comment}
- Compare concepts
- Find optimal concept that is able to meet the requirements
\end{comment}

In this report, a trade-off is performed between five concepts in order to determine what type of Hybrid UAV layout is optimal for the given requirements. From these requirements, the most important ones are a maximum speed of 200 km/h, an endurance of at least one hour, a payload carrying capacity of 10 kg and the vertical take-off and landing. The concepts that are compared are a Tailsitter, a Tandem configuration, a Prandtl boxed wing configuration, an aircraft with tilting rotors and a hybrid quadcopter design. Based on a number of criteria, which in turn were based on the requirements, the concepts are compared to each other. 


It was found that overall, the Tandem concept performed the worst. It especially lacked on flight performance and sustainability. The Tiltrotor concept did not perform well either and lacked on the same points plus production cost. The Prandtl Box concept performed quite well, and scored very well on development risk, reliability and performance. But the design that is expected to meet all requirements the best is the Winged Quadcopter, that performed very well compared to the others on each point. Because of this, the decision has been made to use the Winged Quadcopter to continue the design process.



%Concepts used in TO, criteria
\begin{comment}
- Describe all 5 concepts, refer to where they can be found
- Summarise all criteria
\end{comment}

%Result of TO, elaborate on concept
\begin{comment}
- Explain result and why
- Further elaborate on concept
\end{comment}

















