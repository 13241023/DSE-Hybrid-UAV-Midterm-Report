\chapter{Introduction}

The ultimate goal of this hybrid Unmanned Aerial Vehicle (UAV) project is to design a UAV that combines the capabilities of high velocity horizontal flight with vertical take-off and landing (VTOL) and hovering. The project aims document and present an efficient Hybrid UAV design which can be controlled remotely within visual line of sight (VLOS) with the capability of beyond visual line of sight (BVLOS) control in the future. Derived from the aforementioned goal is the project objective statement (POS) which summarises the goal of the project and the steps required to achieve it in one statement and is characterised as follows.

\nomenclature[A]{UAV}{Unmanned Aerial Vehicle}
\nomenclature[A]{VTOL}{Vertical Take-off and Landing}
\nomenclature[A]{VLOS}{Visual Line of Sight}
\nomenclature[A]{BVLOS}{Beyond Visual Line of Sight}
\nomenclature[A]{POS}{Project Objective Statement}
\nomenclature[A]{MNS}{Mission Needs Statement}

\begin{quote}
\begin{itshape}
Design and optimise a Hybrid UAV that meets requirements and constraints by using project management and systems engineering tools, by a team of 10 students within 11 weeks.
\end{itshape}
\end{quote}

The midterm report serves to document both the trade-off and the analyses necessary to perform it. The required functions of the Hybrid UAV identified and defined in the baseline report form the foundation upon which the requirements and constraints can be determined \cite{baseline}. These functions and requirements follow indirectly from the mission need statement (MNS) which concisely defines what the goal of the design is and is characterised as follows.

\begin{quote}
\begin{itshape}
Carry out both supervised and autonomous monitoring and transport missions, comprising vertical take-off and landing, and sustained high-velocity horizontal flight.
\end{itshape}
\end{quote}

In \autoref{ch:proj_orga} the structure of the group is defined as well as the plan of attack of the whole project. This includes the work flow diagram (WFD), the work breakdown structure (WBS) and the Gantt chart. \autoref{ch:requ} presents the requirements and constraints on the system as well as classifies these requirements as killer, key or driving. In \autoref{ch:concepts} the five concepts that further analysis will be conducted on are presented and \autoref{ch:trad_off} presents the trade-off of the said concepts. In Chapters \ref{ch:perf_analy} to \ref{ch:grou_hand} the analyses of the trade-off criteria are presented. In each of these chapters either a sub-trade-offs is conducted or a grading system is defined. The N2 charts are introduced in \autoref{ch:n2_char} and characterise the interfaces between the subsystems as well as the relationships between subsystems during operation. A life cycle assessment procedure is presented in \autoref{ch:sustaindev} which lays down the framework to assess the sustainability of the whole process. In \autoref{ch:tech_risk_asse} the risk of of each concept not meeting the requirements is assessed and possible mitigation strategies are presented. \autoref{ch:prod_plan} presents an assembly based production plan which, should the UAV go into production, provides information that on the production process. \autoref{sec:ol} on operations and logistics focuses on defining the support facilities that need to be in place as well as the activities that need to be carried out so that the UAV can perform the prescribed missions. In \autoref{ch:layo_and_conf} the internal layout of the main subsystems is displayed along with the communication flow which defines the flow of information, data and commands through the system. Lastly, \autoref{ch:v_and_v} the verification and validation procedures are presented.

\nomenclature[A]{WFD}{Work Flow Diagram}
\nomenclature[A]{WBS}{Work Breakdown Structure}